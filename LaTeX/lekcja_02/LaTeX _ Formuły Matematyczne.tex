\documentclass[a4paper,12pt]{article}
\usepackage[MeX]{polski}
\usepackage[utf8]{inputenc}
\usepackage{algorithmic}

%opening
\title{Wzory matematyczne}
\author{Dasher}
\date{24.10.2017}

\begin{document}
\maketitle

Wykorzystanie formuł matematycznych w LaTeX

\tableofcontents

\section{Geneza}
Matematycy od dawna potrzebowali urządzenia, które pozwoli im pisać skomplikowane wzory. Jedną z opcji został LateX.

\section{Polecenie do wykonania}

\begin{equation}
\label{eg:lim}
\lim_{n \to \infty}
	\sum_{k=1}^n \frac{1}{k^2}
		= \frac{\pi^2}{6}
\end{equation}

\begin{equation}
\label{eg:prod}
\prod_{i=2}^{n=i^2}
	= \frac{\lim^{n \to 4}{(1+\frac{1}{n})^n}}{\sum k(\frac{1}{n})}
\end{equation}

Łatwo równanie \ref{eg:lim} jest doprowadzić do \ref{eg:prod}

\begin{equation}
\label{eg:int}
\int_{2}^{\infty} \frac{1}{\log_{2}x}dx= 
	\frac{1}{x}\sin{x}=1-\cos^2{(x)}
\end{equation}

\begin{equation}
\label{cos}
c'_{ij}=
\left \{
	\begin{array}{c}
		e_{ij}\ {\rm gdy}\ d(x_i) \neq d(x_j) \\
		\phi\ {\rm gdy}\ d(x_i)=d(x_j) \\
	\end{array}
	\right
\end{equation}

\begin{equation}
\label{eg:array}
\left[ \begin{array}{cccc}
a_{11} & a_{12} & \ldots & a_{1K} \\
a_{21} & a_{22} & \ldots & a_{2K} \\
\vdots & \vdots & \ddots & \vdots \\
a_{K1} & a_{K2} & \ldots & a_{KK} \\
\end{array} \right]*
\left[ \begin{array}{c}
x_{1} \\
x_{2} \\
\vdots \\
x_{K} \\
\end{array} \right]=
\left[ \begin{array}{c}
b_{1} \\
b_{2} \\
\vdots \\
b_{K} \\
\end{array} \right]
\end{equation}

\begin{algorithmic}
\STATE{Procedure}
\STATE{Input data}
\STATE{$A' \leftarrow \emptyset$}
\STATE{$iter \leftarrow 0$}
\FOR {i=1, 2, \dots, card\{A\}}
\FOR {j=1, 2, \dots, k}
\STATE{$S^{c_j}(a)=S_{i}^{c_j}(a)$}
\IF {$a \not \in A'$}
\item {$A' \leftarrow \ a$}
\item {$itter \leftarrow itter+1$}
\IF {$itter=fixed\ number\ of\ the\ best\ genes$}
\item {BREAK}
\ENDIF
\ENDIF
\ENDFOR
\IF{$iter=fixed\ number\ of\ the\ best\ genes$}
\item{BREAK}
\ENDIF
\ENDFOR
\RETURN{$A'$}
\end{algorithmic}

\end{document}