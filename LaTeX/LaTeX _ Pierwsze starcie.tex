\documentclass[a4paper,12pt]{article}
\usepackage[MeX]{polski}
\usepackage[utf8]{inputenc}

%opening
\title{Wydział Matematyki i Informatyki Uniwersytetu Warmińsko-Mazurskiego}
\author{Dasher}
\date{17.10.2017}

\begin{document}
\maketitle
Wydział Matematyki i Informatyki Uniwersytetu Warmińsko-Mazurskiego (WMiI) --- wydział Uniwersyteru Warmińsko-Mazurskiego w Olsztynie oferujący studia na dwóch kierunkach:

\begin{itemize}
\item Informatyka
\item Matematyka
\end{itemize}

\newline{}w trybie studiów stacjonarnych i niestacjonarnych. Ponadto oferuje sudia podyplomowe.
\newline{}
\newline{}Wydział zatrudnia 8 profesorów, 14 doktorów habilitowanych, 53 doktorów i 28 magistrów

\tableofcontents

\section{Misja}
Misją Wydziału jest:

\begin{itemize}
\item Kształcenie matematyków zdolnych do udziału w rozwijaniu matematyki i jej stosowania w innych działach wiedzy i w praktyce;
\item Kształcenie nauczycieli matematyki, nauczycieli matematyki z fizyką a także nauczycieli informatyki;
\item Kształcenie profesjonalnych informatyków dla potrzeb gospodarki, administracji, szkolnictwa oraz życia
społecznego;
\item Nauczanie matematyki i jej działów specjalnych jak statystyka matematyczna, ekonometria,
biomatematyka, ekologia matematyczna, metody numeryczne; fizyki a w razie potrzeby i podstaw
informatyki na wszystkich wydziałach UWM.
\end{itemize}

\section{Opis kierunków$^{[1]}$}
Na kierunku Informatyka prowadzone są studia stacjonarne i niestacjonarne:

\begin{itemize}
\item studia pierwszego stopnia – inżynierskie (7 sem.), sp. inżynieria systemów informatycznych, informatyka ogólna
\item studia drugiego stopnia – magisterskie (4 sem.), sp. techniki multimedialne, projektowanie systemów informatycznych i sieci komputerowych
\end{itemize}

Na kierunku Matematyka prowadzone są studia stacjonarne:

\begin{itemize}
\item studia pierwszego stopnia – licencjackie (6 sem.), sp. nauczanie matematyki, matematyka stosowana
\item studia drugiego stopnia – magisterskie (4 sem.), sp. nauczanie matematyki, matematyka stosowana
\end{itemize}

oraz studia niestacjonarne:

\begin{itemize}
\item studia drugiego stopnia – magisterskie (4 sem.), sp. nauczanie matematyki
\end{itemize}

\newline{}Państwowa Komisja Akredytacyjna w dniu 19 marca 2009r. oceniła pozytywnie jakość kształcenia na kierunku Matematyka, natomiast w dniu 12 marca 2015r. oceniła pozytywnie jakość kształcenia na kierunku Informatyka$^{[2]}$.

\end{document}